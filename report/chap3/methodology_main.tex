\chapter{Methodology}
\label{ch:method}

\section{Approach}

%The general approach that's been applied here is to use modelled outputs (derived from empirical data) to identify spatiotemporal correlations between vegetation change and atmospheric variables. The rationale for this is that although it is often difficult to identify whether a (semi-)empirical observation is the result of some physical mechanism(s) or just a chance occurrence, where a relationship holds tightly through space and/or time the latter is much less likely to be the case. Therefore, if a spatiotemporal correlation between two (or more) variables can be identified, it is suggestive that there is some underlying relationship which is not due to chance, especially if the correlation is very strong (for example, if the spatial distribution of some variable has a very particular shape yet coincides almost exactly with the spatial distribution of another variable). Furthermore, we would have even greater confidence that an identified pattern represents a non-spurious relationship if such a pattern were present in data across various disparate regions.
%
%Even if such a suggestive relationship can be identified, the question still remains regarding whether there is a causal relationship (direct or indirect) between the variables, or whether there are confounding factors with the effect of producing the identified relationship. In the context of science, the question of causation can only be raised sensibly relative to a theoretical framework, which in turn must be at least internally consistent and congruent with rigidly established scientific frameworks (such as the laws of physics) - unless there is extraordinary evidence to reject the latter. Evidence of a causal relationship between empirical variables is then equivalent to there being a logically sound (or at least plausible) explanation within that theoretical framework as to why the identified relationship holds, and for which there exists empirical data supporting the proposed explanation.
%
%Where there are multiple such frameworks and discussion of results crosses over multiple frameworks, the "causes" of something should be specified along with the framework which the "causes" are relative to so as to avoid confusion. If there is an insistence upon using only a single framework, the choice is a matter of judgement but historically \citep{kuhn1970}, selecting a framework \textit{for general use} has been weighted upon criteria such as: 
%\begin{enumerate}
%	\item Accurate: "demonstrated agreement with the results of existing experiments and observations" \citep{kuhn1977}
%	\item Consistent: "not only internally or with itself, but also with other currently accepted theories applicable to related aspects of nature" \citep{kuhn1977}
%	\item Broad Scope: "consequences should extend far beyond the particular observations, laws or subtheories it was initially designed to explain" \citep{kuhn1977}
%	\item Simple: "bringing order to phenomena that in its absence would be individually isolated and, as a set, confused" \citep{kuhn1977}
%	\item Fruitful: "disclose new phenomena or previously unnoted relationships among those already known" \citep{kuhn1977}
%\end{enumerate}
%
%(Select framework to analyse a particular problem)
%
%(Mention how this will be relevant as this study area is still not well understood and there are various proposed frameworks)
%
%(Explain choice to plot main statistics of MDP since this is the easiest way to visualise diurnal profiles, but mention plotting by hourly values is also possible)
%
%(Explain 5-year rolling avg of climate indices)
%
%(Explain use of study regions)

(ADD A FLOW CHART HERE)

To identify how vegetation loss may affect (or has historically affected) surface winds, we produced a series of spatial plots using GLASS-satellite derived data and ERA5 reanalysis data. These plots sought to uncover any spatiotemporal correlations between vegetation loss and key atmospheric variables such as wind speed, wind direction and mean sea level pressure. The rationale behind this was that were there to be any concrete spatiotemporal correlations, it would suggest strongly that there is some underlying dynamic between the variables (since a concrete pattern manifesting through both space and time purely by chance is unlikely).

In doing this, we first identified three focus regions which were likely to yields results either due to historically extensive degrees of vegetation change or other unique circumstances such as having a sharp delineation between natural and agricultural land cover. For each of these regions, we then strategically selected two 5-year long historical periods for comparison. The periods were selected in such a fashion so as to control (to the extent possible) for other effects such as atmospheric oscillations which may also affect the key atmospheric variables of interest. Period lengths of 5 years were selected since this averages out to some extent the effect of shorter-term atmospheric fluctuations.

To assist in this selection, we created yearly spatial plots for the 5-year rolling average of the annual difference in mean leaf area index. We also created 5-year rolling average line plots for the climate indices corresponding to each region's relevant (i.e. climate-driving) atmospheric oscillations. From this, we selected a set of study periods for each region based on the following criteria (in order of priority):
\begin{enumerate}
	\item 5-year rolling averages for relevant climate indices were similar
	\item Change in leaf area index between the periods was extensive
	\item Periods cover a similar amount of time spent in La Nina / El Nino (where relevant) and Negative / Positive Indian Ocean Dipole (where relevant)
	\item Monthly values for relevant climate indices over each period display a similar time evolution pattern
\end{enumerate}

We also selected a second set of study periods for each region, for which the 5-year averages of relevant climate indicies had \textit{dissimilar} values. The rationale behind this was that were some relationship to be identified in the analysis between \textit{similar} periods, we could evaluate our confidence in that relationship by seeing whether it holds even under extreme differences in atmospheric conditions.

For each of the these regions and study periods, we then created a series of spatial plots displaying summary statistics for the key atmospheric variables and the differences in these values between periods (other related variables which might affect the behaviour of the key variables were also studied).

Central to these summary statistics was a variable's "mean diurnal profile" over each period, produced using a group-wise average by hour of day for all the data values in that period. This was computed for each grid cell in the data, and was used to study how vegetation change might affect the diurnal variations in key variables. Because of the difficulty in visualising the temporal variability in spatial data, we created plots for the hour of maximum, hour of minimum, maximum, minimum, mean and range of these mean diurnal profile values. We also created spatial plots for the mean diurnal profile value at each hour of the day, to analyse cotemporaneous diurnal evolution of different variables.

In addition to this, we created spatial plots for other distributional properties over each period (and the difference in these values between periods) such as the wind speed at 100 m above surface such as standard deviation, gross capacity factor for a typical wind turbine with 100 m hub height, empirical fits for the Weibull parameters, and the expected rate of exceeding 42.5 m/s (the typical speed which a turbine can withstand for 10 minutes).

Spatiotemporal correlations were identified by visual inspection as this was deemed more appropriate than a rigid statistic metric, since the latter will have to be computed upon gridded data and hence may miss spatial correlations between variables which are present but manifest slightly offset from each other by a few grid cells (and it is non-trivial to systematically correct for all the different spatial variations by which two variables can be slightly offset from each other).

\section{Focus regions}

\subsection{Western Australia}

We selected the south-west part of Western Australia (from 114$^\circ$E to 124$^\circ$E longitude and 36$^\circ$S to 6$^\circ$S latitude; SEE FIGURE) for investigation because previous studies have suggested significantly different atmospheric conditions on each side of the State Boundary Fence of Western Australia, which itself sharply delineates native vegetation (on the eastern side of the fence) from agricultural land (on the western side of the fence). 

\subsection{Central America}

\subsection{South America}

\section{Study periods}

\subsection{Annual difference in mean leaf area index}

\subsubsection{Western Australia}

\subsubsection{Central America}

\subsubsection{South America}

\subsection{Climate indices}

\section{Variables analysed}

\section{Statistical summaries}

- Describe how the plots were made
- Describe the high level functions designed and used
- Describe what we are looking for in each column of the plots

\subsection{Mean diurnal profile statistics}

- mention time zones

\footnote{These averages were produced mostly by computing the mean of monthly averaged by hour of day data, which is roughly equivalent to the mean of all hourly data values. Vector variables were computed using hourly data values. Reasons for these choices are discussed in Appendix \ref{app:commutativity}.}

\subsection{Mean diurnal profile values}

\subsection{Wind speed distribution}

\section{Datasets}

\subsection{ERA5 reanalysis data}

On single levels, Reanalysis data for atmospheric parameters

\subsection{GLASS satellite-derived data}

Long-term satellite-derived products for land surface variables

\subsection{NOAA climate indices}

\section{Software}

\section{Reproducibility}

This section should include a recipe of what you did (explain what you have done so if someone wants to reproduce the experiment, they can).  A flow chart is typically helpful.  Also, make sure to define all software that you used including version numbers and OS.  Should also include a description of statistical methods used (if any).\footnote{For more information see: \url{http://rc.rcjournal.com/content/49/10/1229.short}}

\blindtext

\section*{Summary}
\blindtext