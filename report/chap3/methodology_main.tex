\chapter{Methodology}

\section{Approach}

The general approach that's been applied here is to use modelled outputs (derived from empirical data) to identify spatiotemporal correlations between vegetation change and atmospheric variables. The rationale for this is that although it is often difficult to identify whether a (semi-)empirical observation is the result of some physical mechanism(s) or just a chance occurrence, where a relationship holds tightly through space and/or time the latter is much less likely to be the case. Therefore, if a spatiotemporal correlation between two (or more) variables can be identified, it is suggestive that there is some underlying relationship which is not due to chance, especially if the correlation is very strong (for example, if the spatial distribution of some variable has a very particular shape yet coincides almost exactly with the spatial distribution of another variable). Furthermore, we would have even greater confidence that an identified pattern represents a non-spurious relationship if such a pattern were present in data across various disparate regions.

Even if such a suggestive relationship can be identified, the question still remains regarding whether there is a causal relationship (direct or indirect) between the variables, or whether there are confounding factors with the effect of producing the identified relationship. In the context of science, the question of causation can only be raised sensibly relative to a theoretical framework, which in turn must be at least internally consistent and congruent with rigidly established scientific frameworks (such as the laws of physics) - unless there is extraordinary evidence to reject the latter. Evidence of a causal relationship between empirical variables is then equivalent to there being a logically sound (or at least plausible) explanation within that theoretical framework as to why the identified relationship holds, and for which there exists empirical data supporting the proposed explanation.

Where there are multiple such frameworks and discussion of results crosses over multiple frameworks, the "causes" of something should be specified along with the framework which the "causes" are relative to so as to avoid confusion. If there is an insistence upon using only a single framework, the choice is a matter of judgement but historically \citep{kuhn1970}, selecting a framework \textit{for general use} has been weighted upon criteria such as: 
\begin{enumerate}
	\item Accurate: "demonstrated agreement with the results of existing experiments and observations" \citep{kuhn1977}
	\item Consistent: "not only internally or with itself, but also with other currently accepted theories applicable to related aspects of nature" \citep{kuhn1977}
	\item Broad Scope: "consequences should extend far beyond the particular observations, laws or subtheories it was initially designed to explain" \citep{kuhn1977}
	\item Simple: "bringing order to phenomena that in its absence would be individually isolated and, as a set, confused" \citep{kuhn1977}
	\item Fruitful: "disclose new phenomena or previously unnoted relationships among those already known" \citep{kuhn1977}
\end{enumerate}



(Select framework to analyse a particular problem)

(Mention how this will be relevant as this study area is still not well understood and there are various proposed frameworks)

(Explain choice to plot main statistics of MDP since this is the easiest way to visualise diurnal profiles, but mention plotting by hourly values is also possible)

(Explain 5-year rolling avg of climate indices)

(Explain use of study regions)

To identify how vegetation loss may affect (or has historically affected) surface winds, we produced a series of spatial plots using ERA5 reanalysis data which seek to uncover any spatiotemporal correlations between vegetation loss and key atmospheric variables such as wind speed, wind direction and mean sea level pressure. The rationale behind this was that were there to be any concrete spatiotemporal correlations,
it would suggest strongly that there is some underlying dynamic between the variables (since a concrete pattern manifesting through both space and time purely by chance is unlikely).

In doing this, we first identified three focus regions which were likely to yields results either due to historically extensive degrees of vegetation change or other unique circumstances. For each of these regions, we then strategically selected two 5-year historical periods for comparison. The periods were selected in such a fashion so as to control (to the extent possible) for other effects such as atmospheric oscillations which may also affect the key atmospheric variables of interest. Period lengths of 5 years were selected since this averages out to some extent the effect of shorter-term fluctuations.

To assist in this selection, yearly spatial plots for the 5-year rolling average of the annual difference in mean leaf area index were created. Climate indices for each region's relevant (i.e. climate-driving) atmospheric oscillations were also obtained, and the 5-year rolling averages for these were plotted. The periods were selected based on the following criteria (in order of priority):
1. 5-year rolling averages for relevant climate indices were similar
2. Change in leaf area index between the periods was extensive
3. Monthly values for relevant climate indices over each period display a similar pattern 

\subsection{Focus regions}

\subsubsection{Central America}

\subsubsection{South America}

\subsubsection{Western Australia}

\subsection{Statistical metrics}

\subsubsection{Mean diurnal profile climatology}

\subsubsection{Weibull parameters}

\section{Reproducibility}

\section{Datasets}

\subsection{Reanalysis data for atmospheric variables}

\subsection{Long-term satellite-derived products for land surface variables}

\section{Software}

This section should include a recipe of what you did (explain what you have done so if someone wants to reproduce the experiment, they can).  A flow chart is typically helpful.  Also, make sure to define all software that you used including version numbers and OS.  Should also include a description of statistical methods used (if any).\footnote{For more information see: \url{http://rc.rcjournal.com/content/49/10/1229.short}}

\blindtext

\section*{Summary}
\blindtext