\chapter{Conclusions and Summary}
\label{ch:conclusions}

\begin{itemize}
	\item The results present strong evidence that the clearing of native vegetation for agriculture in southwest Western Australia has changed atmospheric energy conversions, in part due to changes in the surface energy balance.
	\item The results present strong evidence that land cover produces local modulations upon background synoptic changes, with typically dampened fluctuations.
	\item The results present moderate but indirect evidence (from water vapour flows) that surface wind circulation patterns will be affected beyond just effects from roughness length changes.
	\item The results present moderate evidence in support of the existence of condensation-induced atmospheric dynamics. This includes native vegetation modulation of atmospheric and moisture convergences leading to altered mass and moisture transport respectively.
	\item The results present moderate evidence that historical land cover change in southwest Western Australia has led to decreases in coastal temperatures which has weakened the summer daytime sea breeze. This has likely caused the observed inland rainfall loss, decreased inland moisture convergence (possibly indicting wind convergence), and an increased coastal flood risk. Meanwhile, the effect of future land cover change is highly variable depending on what type of change takes place.
	\item For wind energy operation, the implications for this are decreased generation and operating revenue during peak season, increased risk of infrastructure damage from floods, and increased risk embodied from uncertainty regarding what future land cover change will take place. A weakened sea breeze also implies less cool relief to the urban centre of Perth during summer daytime, so a decrease in supply coincides with an increase in demand (see Section~\ref{ssec:implications}).
	\item A conceptual framework for understanding how land cover affects surface winds was created, along with a heuristical procedure for qualitatively assessing the likely effect of future land cover change on circulation patterns.
	\item A series of purpose-built Python programs was developed for analysing how vegetation change affects the diurnal and seasonal variations of an arbitrary variable from the \ac{ERA5} dataset (see Appendix~\ref{app:code}).
\end{itemize}

%\enlargethispage{\baselineskip} % so you do not get a single line in another page