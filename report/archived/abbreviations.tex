%% will only print what is used ... useful.
%% also acronyms are clickable, which is awesome

\chapter{List of Abbreviations} %% \chapter*{List of Abbreviations} not to appear in LoC
\markboth{List of Abbreviations}{List of Abbreviations}
               
\begin{acronym}\itemsep-20pt\parsep-20pt %% if you remove these spacing params this list becomes huge!
\acro{LCC}{Land Cover Change}
\acro{LULCC}{Land Use and Land Cover Change}
\acro{NSWS}{Near Surface Wind Speed(s)}
\acro{SWS}{Surface Wind Speed(s)}
\acro{SOA}{Secondary Organic Aerosols}
\acro{VOC}{Volatile Organic Compounds}
\acro{CCN}{Cloud Condensation Nuclei}
\acro{AWS}{Automatic Weather Station(s)}
\acro{CIAD}{Condensation-Induced Atmospheric Dynamics}
\acro{LCL}{Lifted Condensation Level}
\acro{RH}{Relative Humidity}
\acro{SST}{Sea Surface Temperature}
\acro{DJF}{December-January-February}
\acro{MAM}{March-April-May}
\acro{JJA}{June-July-August}
\acro{SON}{September-October-November}
\acro{LT}{Local Time}

\acro{SBFWA}{State Boundary Fence of Western Australia}
\acro{WA}{Western Australia}
\acro{CA}{Central America}
\acro{NB}{Northern Brazil}

\acro{MDP}{Mean Diurnal Profile}
\acro{WSD}{Wind Speed Distribution}
\acro{LSE}{Land Surface Elevation}
\acro{SSGO}{Slope of Sub-Gridscale Orography}
\acro{LAI}{Leaf Area Index}
\acro{MLAI}{Mean Leaf Area Index}
\acro{FAPAR}{Fraction of Photosynthetically Absorbed Radiation}
\acro{MFAPAR}{Mean Fraction of Photosynthetically Absorbed Radiation}
\acro{BLH}{Boundary Layer Height}
\acro{CAPE}{Convective Available Potential Energy}
\acro{CBH}{Cloud Base Height}
\acro{FA}{Forecast Albedo}
\acro{MSLP}{Mean Sea Level Pressure}
\acro{NAC}{Net Atmospheric Condensation}
\acro{NSE}{Net Surface Evaporation}
\acro{SLHF}{Surface Latent Heat Flux}
\acro{SSHF}{Surface Sensible Heat Flux}
\acro{T2}{Temperature at 2 m Above Surface}
\acro{TCC}{Total Cloud Cover}
\acro{TCCLW}{Total Column Cloud Liquid Water}
\acro{TCWV}{Total Column Water Vapour}
\acro{U10}{Zonal Component of Wind Velocity at 10 m Above Surface}
\acro{U100}{Zonal Component of Wind Velocity at 100 m Above Surface}
\acro{V10}{Meridional Component of Wind Velocity at 10 m Above Surface}
\acro{V100}{Meridional Component of Wind Velocity at 100 m Above Surface}
\acro{VIDMF}{Vertical Integral of Divergence of Moisture Flux}
\acro{VIEC}{Vertical Integral of Energy Conversion}
\acro{VIKE}{Vertical Integral of Kinetic Energy}
\acro{VIPILE}{Vertical Integral of Potential, Internal and Latent Energy}
\acro{WS10}{Wind Speed at 10 m Above Surface}
\acro{WS100}{Wind Speed at 100 m Above Surface}
\acro{WV10}{Wind Velocity at 10 m Above Surface}
\acro{WV100}{Wind Velocity at 100 m Above Surface}

\acro{dBLH}{Hourly Change in Boundary Layer Height}
\acro{dCAPE}{Hourly Change in Convective Available Potential Energy}
\acro{dCBH}{Hourly Change in Cloud Base Height}
\acro{dFA}{Hourly Change in Forecast Albedo}
\acro{dMSLP}{Hourly Change in Mean Sea Level Pressure}
\acro{dNAC}{Hourly Change in Net Atmospheric Condensation}
\acro{dNSE}{Hourly Change in Net Surface Evaporation}
\acro{dSLHF}{Hourly Change in Surface Latent Heat Flux}
\acro{dSSHF}{Hourly Change in Surface Sensible Heat Flux}
\acro{dT2}{Hourly Change in Temperature at 2 m Above Surface}
\acro{dTCC}{Hourly Change in Total Cloud Cover}
\acro{dTCCLW}{Hourly Change in Total Column Cloud Liquid Water}
\acro{dTCWV}{Hourly Change in Total Column Water Vapour}
\acro{dU10}{Hourly Change in Zonal Component of Wind Velocity at 10 m Above Surface}
\acro{dU100}{Hourly Change in Zonal Component of Wind Velocity at 100 m Above Surface}
\acro{dV10}{Hourly Change in Meridional Component of Wind Velocity at 10 m Above Surface}
\acro{V100}{Hourly Change in Meridional Component of Wind Velocity at 100 m Above Surface}
\acro{dVIDMF}{Hourly Change in Vertical Integral of Divergence of Moisture Flux}
\acro{dVIEC}{Hourly Change in Vertical Integral of Energy Conversion}
\acro{dVIKE}{Hourly Change in Vertical Integral of Kinetic Energy}
\acro{dVIPILE}{Hourly Change in Vertical Integral of Potential, Internal and Latent Energy}
\acro{dWS10}{Hourly Change in Wind Speed at 10 m Above Surface}
\acro{dWS100}{Hourly Change in Wind Speed at 100 m Above Surface}
\acro{dWV10}{Hourly Change in Wind Velocity at 10 m Above Surface}
\acro{dWV100}{Hourly Change in Wind Velocity at 100 m Above Surface}

\acro{C10}{Scale Parameter of Wind Speed Weibull Distribution at 10 m Above Surface}
\acro{C100}{Scale Parameter of Wind Speed Weibull Distribution at 100 m Above Surface}
\acro{K10}{Shape Parameter of Wind Speed Weibull Distribution at 10 m Above Surface}
\acro{K100}{Shape Parameter of Wind Speed Weibull Distribution at 100 m Above Surface}
\acro{EROE100}{Expected Rate of Exceeding 42.5 m/s for Wind Speed Weibull Distribution at 100 m Above Surface}
\acro{TGCF100}{Gross Capacity Factor for a Typical 2.5 MW Turbine at 100 m Above Surface given the Wind Speed Weibull Distribution at 100 m Above Surface}

\acro{GE}{General Electric}
\acro{GW}{Goldwind}

\acro{AMOI}{Atlantic Multidecadal Oscillation Index}
\acro{AMO}{Atlantic Multidecadal Oscillation}
\acro{PDOI}{Pacific Decadal Oscillation Index}
\acro{PDO}{Pacific Decadal Oscillation}
\acro{ONI}{Oceanic Nino Index}
\acro{ENSO}{El Nino-Southern Oscillation}
\acro{DMI}{Dipole Mode Index}
\acro{IOD}{Indian Ocean Dipole}
\acro{AAOI}{Antarctic Oscillation Index}
\acro{AAO}{Antarctic Oscillation}
\acro{SAM}{Southern Annular Mode}
\acro{AOI}{Arctic Oscillation Index}
\acro{AO}{Arctic Oscillation}
\acro{NAM}{Northern Annular Mode}
\acro{NAOI}{North Atlantic Oscillation Index}
\acro{NAO}{North Atlantic Oscillation}
\acro{EPOI}{Eastern Pacific Oscillation Index}
\acro{EPO}{Eastern Pacific Oscillation}
\acro{NPO}{Northern Pacific Oscillation}

\acro{AVHRR}{Advanced Very High Resolution Radiometer}
\acro{MODIS}{Moderate Resolution Imaging Spectroradiometer}
\acro{UAV}{Unmanned Aerial Vehicle}
\acro{HPC}{High-Performance Computing}

\acro{UNSW}{University of New South Wales}
\acro{JMA}{Japanese Meteorological Agency}
\acro{BOM}{(Australian) Bureau of Meteorology}
\acro{ECMWF}{European Centre for Medium-Range Weather Forecasts}
\acro{GLASS}{Global Land Surface Satellite}
\acro{ERA5}{ECMWF Reanalysis v5}
\acro{NOAA}{National Oceanic and Atmospheric Administration}
\acro{PSL}{Physical Sciences Laboratory}
\acro{CPC}{Climate Prediction Center}
\acro{CDR}{Climate Data Record}
\acro{NDVI}{Normalised Difference Vegetation Index}

\acro{CDMA}{Code Division Multiple Access}
\acro{GSM}{Global System for Mobile communication}
\acro{NAD+}[NAD\textsuperscript{+}]{Nicotinamide Adenine Dinucleotide}
\acro{NUA}{Not Used Acronym}
\acro{TDMA}{Time Division Multiple Access}
\acro{UA}{Used Acronym}
\acro{lox}[\ensuremath{LOX}]{Liquid Oxygen}
\acro{lh2}[\ensuremath{LH_2}]{Liquid Hydrogen}
\acro{IC}{Integrated Circuit}
\acro{BUT}{Block Under Test}
\acrodefplural{BUT}{Blocks Under Test}    
\end{acronym}
